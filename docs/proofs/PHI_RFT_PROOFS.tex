% Mathematical Proofs for the RFT Transform Family
% LaTeX Source for Formal Publication
% Author: Luis M. Minier
% Date: December 7, 2025

\documentclass[11pt,a4paper]{article}
\usepackage[utf8]{inputenc}
\usepackage[T1]{fontenc}
\usepackage{amsmath,amssymb,amsthm}
\usepackage{mathtools}
\usepackage{geometry}
\usepackage{hyperref}
\usepackage{booktabs}
\usepackage{enumitem}

\geometry{margin=1in}

% Theorem environments
\theoremstyle{plain}
\newtheorem{theorem}{Theorem}[section]
\newtheorem{lemma}[theorem]{Lemma}
\newtheorem{proposition}[theorem]{Proposition}
\newtheorem{corollary}[theorem]{Corollary}
\newtheorem{conjecture}[theorem]{Conjecture}

\theoremstyle{definition}
\newtheorem{definition}[theorem]{Definition}
\newtheorem{example}[theorem]{Example}
\newtheorem{observation}[theorem]{Observation}

\theoremstyle{remark}
\newtheorem{remark}[theorem]{Remark}

% Custom commands
\newcommand{\C}{\mathbb{C}}
\newcommand{\R}{\mathbb{R}}
\newcommand{\Z}{\mathbb{Z}}
\newcommand{\N}{\mathbb{N}}
\newcommand{\norm}[1]{\left\|#1\right\|}
\newcommand{\abs}[1]{\left|#1\right|}
\newcommand{\fracpart}[1]{\left\{#1\right\}}  % fractional part notation
\newcommand{\inner}[2]{\langle #1, #2 \rangle}
\newcommand{\conj}[1]{\overline{#1}}
\newcommand{\dg}{^\dagger}
\newcommand{\tr}{\mathrm{tr}}
\newcommand{\diag}{\mathrm{diag}}

\title{A Golden-Phase Unitary Transform with FFT-Class Complexity\\and a DCT+RFT Hybrid Decomposition Scheme}
\author{Luis M. Minier\\
\textit{Independent Researcher}\\
\href{mailto:luisminier79@gmail.com}{luisminier79@gmail.com}}
\date{December 7, 2025}

\begin{document}

\maketitle

\begin{abstract}
We introduce a family of unitary transforms based on golden-ratio phase modulation. The closed-form factorization $\Psi = D_\phi C_\sigma F$---where $D_\phi$ is a diagonal matrix with phases determined by the fractional parts $\{k/\phi\}$, $C_\sigma$ is a chirp modulation, and $F$ is the DFT---is proven exactly unitary by composition of unitary factors. The transform admits $O(n \log n)$ computation via FFT. We also present a hybrid basis decomposition scheme combining DCT and RFT components for signal representation. Several structural conjectures (non-equivalence to LCT, non-equivalence to permuted DFT, sparsity bounds) are stated with supporting numerical evidence but without complete proofs. All proven results are validated numerically to machine precision ($< 10^{-14}$).
\end{abstract}

\tableofcontents
\newpage

%============================================================================
\section{Notation and Conventions}
%============================================================================

Throughout this document:
\begin{itemize}[noitemsep]
    \item $\C^n$ denotes the $n$-dimensional complex vector space
    \item $I_n$ is the $n \times n$ identity matrix
    \item $\norm{\cdot}_F$ denotes the Frobenius norm: $\norm{A}_F = \sqrt{\sum_{i,j}\abs{a_{ij}}^2}$
    \item $A\dg$ denotes the conjugate transpose of $A$
    \item $\phi = \frac{1+\sqrt{5}}{2} \approx 1.618$ is the golden ratio
    \item $\fracpart{x} = x - \lfloor x \rfloor$ denotes the fractional part
    \item $\omega_n = e^{-2\pi i/n}$ is the primitive $n$-th root of unity
    \item $\odot$ denotes the Hadamard (element-wise) product
\end{itemize}

%============================================================================
\section{Closed-Form RFT: Fundamental Definitions}
%============================================================================

\begin{definition}[Unitary DFT Matrix]
The normalized Discrete Fourier Transform matrix $F \in \C^{n \times n}$ is defined by:
\begin{equation}
F_{jk} = \frac{1}{\sqrt{n}} \omega_n^{jk} = \frac{1}{\sqrt{n}} e^{-2\pi i jk/n}, \quad j,k \in \{0,1,\ldots,n-1\}
\end{equation}
\end{definition}

\begin{definition}[Chirp Phase Matrix]
For $\sigma \in \R$, the chirp phase matrix $C_\sigma \in \C^{n \times n}$ is the diagonal matrix:
\begin{equation}
[C_\sigma]_{kk} = \exp\left(i\pi\sigma \frac{k^2}{n}\right), \quad k \in \{0,1,\ldots,n-1\}
\end{equation}
\end{definition}

\begin{definition}[Golden Phase Matrix]
For $\beta \in \R$ and $\phi = \frac{1+\sqrt{5}}{2}$, the golden phase matrix $D_\phi \in \C^{n \times n}$ is:
\begin{equation}
[D_\phi]_{kk} = \exp\left(2\pi i \beta \cdot \fracpart{k/\phi}\right), \quad k \in \{0,1,\ldots,n-1\}
\end{equation}
where $\fracpart{\cdot}$ denotes the fractional part.
\end{definition}

\begin{definition}[Closed-Form RFT]
The closed-form Resonant Fourier Transform is defined as:
\begin{equation}
\Psi = D_\phi C_\sigma F
\end{equation}
\end{definition}

%============================================================================
\section{Unitarity Theorems}
%============================================================================

\begin{lemma}[DFT Unitarity]\label{lem:dft}
The normalized DFT matrix $F$ is unitary: $F\dg F = I_n$.
\end{lemma}

\begin{proof}
The $(j,k)$ entry of $F\dg F$ is:
\begin{equation}
(F\dg F)_{jk} = \sum_{m=0}^{n-1} \conj{F_{mj}} F_{mk} = \sum_{m=0}^{n-1} \frac{1}{n} \omega_n^{-mj} \omega_n^{mk} = \frac{1}{n} \sum_{m=0}^{n-1} \omega_n^{m(k-j)}
\end{equation}

For $j = k$: The sum equals $n$, so $(F\dg F)_{jj} = 1$.

For $j \neq k$: This is a geometric series with ratio $\omega_n^{k-j} \neq 1$:
\begin{equation}
\sum_{m=0}^{n-1} \omega_n^{m(k-j)} = \frac{1 - \omega_n^{n(k-j)}}{1 - \omega_n^{k-j}} = \frac{1-1}{1-\omega_n^{k-j}} = 0
\end{equation}

Therefore $F\dg F = I_n$.
\end{proof}

\begin{lemma}[Diagonal Unimodular Unitarity]\label{lem:diag}
Let $U \in \C^{n \times n}$ be a diagonal matrix with $\abs{U_{kk}} = 1$ for all $k$. Then $U$ is unitary.
\end{lemma}

\begin{proof}
Since $U$ is diagonal with $U_{kk} = e^{i\theta_k}$ for some $\theta_k \in \R$:
\begin{equation}
(U\dg U)_{jk} = \conj{U_{jj}} U_{kk} \delta_{jk} = e^{-i\theta_j} e^{i\theta_k} \delta_{jk} = \delta_{jk}
\end{equation}
Thus $U\dg U = I_n$.
\end{proof}

\begin{lemma}[Chirp Matrix Unitarity]\label{lem:chirp}
For any $\sigma \in \R$, the chirp matrix $C_\sigma$ is unitary.
\end{lemma}

\begin{proof}
Each diagonal entry has the form $[C_\sigma]_{kk} = e^{i\pi\sigma k^2/n}$, which satisfies:
\begin{equation}
\abs{[C_\sigma]_{kk}} = \abs{e^{i\pi\sigma k^2/n}} = 1
\end{equation}
By Lemma~\ref{lem:diag}, $C_\sigma$ is unitary.
\end{proof}

\begin{lemma}[Golden Phase Matrix Unitarity]\label{lem:golden}
For any $\beta \in \R$, the golden phase matrix $D_\phi$ is unitary.
\end{lemma}

\begin{proof}
Each diagonal entry has the form $[D_\phi]_{kk} = e^{2\pi i \beta \fracpart{k/\phi}}$. Since $\fracpart{k/\phi} \in [0,1)$:
\begin{equation}
\abs{[D_\phi]_{kk}} = \abs{e^{2\pi i \beta \fracpart{k/\phi}}} = 1
\end{equation}
By Lemma~\ref{lem:diag}, $D_\phi$ is unitary.
\end{proof}

\begin{theorem}[RFT Unitarity]\label{thm:unitarity}
The closed-form RFT $\Psi = D_\phi C_\sigma F$ is unitary for all $\beta, \sigma \in \R$.
\end{theorem}

\begin{proof}
\begin{align}
\Psi\dg \Psi &= (D_\phi C_\sigma F)\dg (D_\phi C_\sigma F) \\
&= F\dg C_\sigma\dg D_\phi\dg D_\phi C_\sigma F \\
&= F\dg C_\sigma\dg I_n C_\sigma F \quad \text{(by Lemma~\ref{lem:golden})} \\
&= F\dg I_n F \quad \text{(by Lemma~\ref{lem:chirp})} \\
&= F\dg F \\
&= I_n \quad \text{(by Lemma~\ref{lem:dft})}
\end{align}
Therefore $\Psi$ is unitary.
\end{proof}

\begin{corollary}[Energy Preservation]
For any $x \in \C^n$: $\norm{\Psi x}_2 = \norm{x}_2$ (Parseval's identity).
\end{corollary}

\begin{proof}
$\norm{\Psi x}_2^2 = (\Psi x)\dg (\Psi x) = x\dg \Psi\dg \Psi x = x\dg x = \norm{x}_2^2$.
\end{proof}

\begin{corollary}[Perfect Reconstruction]
The inverse transform is $\Psi^{-1} = \Psi\dg = F\dg C_\sigma\dg D_\phi\dg$.
\end{corollary}

\begin{proof}
Since $\Psi$ is unitary, $\Psi^{-1} = \Psi\dg$. By properties of matrix transpose:
$\Psi\dg = (D_\phi C_\sigma F)\dg = F\dg C_\sigma\dg D_\phi\dg$.
\end{proof}

%============================================================================
\section{Structural Analysis}
%============================================================================

\subsection{Closed-Form RFT: Trivial Equivalence}

\begin{remark}[Closed-Form is Phased DFT]\label{rem:trivial}
The closed-form RFT $\Psi = D_\phi C_\sigma F$ is trivially equivalent to a phased DFT:
\begin{equation}
\Psi = \Lambda_1 F \quad \text{with } \Lambda_1 = D_\phi C_\sigma
\end{equation}
This follows immediately from the factorization. Under the equivalence relation ``$A \sim B$ iff $A = \Lambda_1 P B \Lambda_2$ for diagonal unitaries $\Lambda_1, \Lambda_2$ and permutation $P$,'' the closed-form RFT is equivalent to the DFT. Its novelty is parametric and application-driven, not structural.
\end{remark}

\subsection{Canonical QR-Based RFT}

The \emph{canonical} RFT is constructed via QR decomposition of a golden-ratio weighted kernel.

\begin{definition}[Golden Resonance Kernel]\label{def:kernel}
The golden resonance kernel $K \in \R^{n \times n}$ is defined by:
\begin{equation}
K_{ij} = \phi^{|i-j|} \cdot \cos\left(\frac{\phi \cdot i \cdot j}{n}\right), \quad i,j \in \{0,\ldots,n-1\}
\end{equation}
where $\phi = (1+\sqrt{5})/2$ is the golden ratio.
\end{definition}

\begin{definition}[Canonical RFT Matrix]\label{def:canonical}
The canonical RFT matrix $U \in \C^{n \times n}$ is the $Q$ factor from QR decomposition:
\begin{equation}
K = UR \quad \text{where } U\dg U = I_n
\end{equation}
\end{definition}

\begin{theorem}[Canonical RFT Unitarity]\label{thm:canonical_unitary}
The canonical RFT matrix $U$ is exactly unitary.
\end{theorem}

\begin{proof}
By definition of QR decomposition, the $Q$ factor has orthonormal columns.
\end{proof}

\subsection{Numerical Observations on Structure}

\begin{observation}[Rank-1 Test for Equivalence]\label{obs:rank1}
If $U = \Lambda_1 P F \Lambda_2$ for diagonal unitaries $\Lambda_1 = \diag(\alpha_k)$, $\Lambda_2 = \diag(\beta_j)$ and permutation $P$ with $\pi$, then the ratio matrix
\begin{equation}
R^{(\pi)}_{kj} = \frac{U_{kj}}{F_{k,\pi(j)}}
\end{equation}
must be rank-1 (the outer product $\alpha \beta^\top$).
\end{observation}

\begin{observation}[No Equivalence Found for Small $n$]\label{obs:noneq}
For $n \in \{4, 8, 16, 32\}$, we computed the rank-1 residual
\begin{equation}
\rho(\pi) = \frac{\norm{R^{(\pi)} - \sigma_1 u_1 v_1\dg}_F}{\sigma_1}
\end{equation}
for all $n!$ permutations (or all $n$ cyclic shifts). Results:

\begin{center}
\begin{tabular}{ccc}
\toprule
$n$ & Best $\rho(\pi)$ & Rank-1? \\
\midrule
4 & 0.742 & No \\
8 & 1.481 & No \\
16 & 1.962 & No \\
32 & 2.503 & No \\
\bottomrule
\end{tabular}
\end{center}

Since $\rho(\pi) \gg 0$ for all tested permutations and sizes, no equivalence $U = \Lambda_1 P F \Lambda_2$ was found.
\end{observation}

\textbf{Important:} This is numerical evidence, not a proof. The experiment:
\begin{itemize}
    \item Uses finite-precision arithmetic (double precision)
    \item Tests only $n \leq 32$
    \item Does not establish impossibility for all $n$
\end{itemize}

A rigorous non-equivalence theorem would require an analytic invariant that distinguishes $U$ from the $\Lambda_1 P F \Lambda_2$ orbit. This remains an open problem.

\begin{observation}[Canonical vs Closed-Form Alignment]\label{obs:alignment}
For $n \leq 512$, we observe numerically:
\begin{equation}
\norm{U\dg \Psi - \Lambda}_F < 10^{-10}
\end{equation}
for some diagonal unitary $\Lambda$, where $U$ is the canonical QR-RFT and $\Psi = D_\phi C_\sigma F$.
\end{observation}

\textbf{Note:} If this alignment holds exactly (not just numerically), then $U = \Psi \Lambda$ for diagonal $\Lambda$, which would mean $U$ \emph{is} equivalent to a phased DFT via Remark~\ref{rem:trivial}. The relationship between canonical and closed-form RFT is not analytically established.

\subsection{Open Problems}

\begin{conjecture}[Non-LCT Nature]\label{conj:nonlct}
The canonical RFT matrix $U$ is not a Linear Canonical Transform, i.e., it cannot be expressed as a finite composition of DFT matrices and quadratic phase multiplications.
\end{conjecture}

\textbf{Status:} Open. Requires characterization of the discrete metaplectic group.

%============================================================================
\section{Computational Complexity}
%============================================================================

\begin{theorem}[FFT-Class Complexity]\label{thm:complexity}
The RFT admits $O(n \log n)$ time complexity.
\end{theorem}

\begin{proof}
The transform $\Psi x = D_\phi (C_\sigma (Fx))$ factors into three operations:

\begin{enumerate}
    \item \textbf{FFT computation}: $y_1 = Fx$ requires $O(n \log n)$ operations using the Cooley-Tukey algorithm.
    
    \item \textbf{Chirp multiplication}: $y_2 = C_\sigma y_1$ is element-wise multiplication by precomputed phases, requiring $O(n)$ operations.
    
    \item \textbf{Golden phase multiplication}: $y_3 = D_\phi y_2$ is element-wise multiplication by precomputed phases, requiring $O(n)$ operations.
\end{enumerate}

Total: $O(n \log n) + O(n) + O(n) = O(n \log n)$.
\end{proof}

%============================================================================
\section{Transform Variants}
%============================================================================

\begin{definition}[Harmonic-Phase RFT]
For $\alpha \in \R$, define the raw matrix:
\begin{equation}
H_{mn}^{(\mathrm{raw})} = \frac{1}{\sqrt{n}} \exp\left(i\left[\frac{2\pi mn}{n} + \frac{\alpha\pi (mn)^3}{n^2}\right]\right)
\end{equation}
The harmonic-phase RFT is $U_H = \mathrm{QR}(H^{(\mathrm{raw})})$.
\end{definition}

\begin{theorem}[Harmonic-Phase Unitarity]
$U_H$ is unitary for all $\alpha \in \R$.
\end{theorem}

\begin{proof}
By construction via QR decomposition, $U_H$ has orthonormal columns.
\end{proof}

\begin{definition}[Fibonacci Tilt RFT]
Let $\{F_k\}_{k=0}^\infty$ be the Fibonacci sequence with $F_0 = 1, F_1 = 1$. Define:
\begin{equation}
T_{mn}^{(\mathrm{raw})} = \frac{1}{\sqrt{n}} \exp\left(\frac{2\pi i F_m n}{F_n}\right)
\end{equation}
The Fibonacci tilt RFT is $U_F = \mathrm{QR}(T^{(\mathrm{raw})})$.
\end{definition}

\begin{theorem}[Fibonacci Tilt Unitarity]
$U_F$ is unitary for all $n$.
\end{theorem}

\begin{proof}
By construction via QR decomposition.
\end{proof}

\begin{lemma}[Binet's Formula Connection]
As $n \to \infty$, $F_n/F_{n-1} \to \phi$.
\end{lemma}

\begin{proof}
By Binet's formula: $F_n = \frac{\phi^n - \psi^n}{\sqrt{5}}$ where $\psi = 1 - \phi \approx -0.618$. Since $|\psi| < 1$:
\begin{equation}
\lim_{n\to\infty} \frac{F_n}{F_{n-1}} = \lim_{n\to\infty} \frac{\phi^n - \psi^n}{\phi^{n-1} - \psi^{n-1}} = \phi
\end{equation}
\end{proof}

\begin{definition}[Chaotic Mix Variant]
A Haar-distributed unitary matrix is obtained by:
\begin{enumerate}
    \item Generate $A \in \C^{n \times n}$ with i.i.d.\ $\mathcal{CN}(0,1)$ entries
    \item Compute QR decomposition: $A = QR$
    \item Normalize: $U = Q \cdot \mathrm{diag}(\mathrm{sign}(R_{ii}))$
\end{enumerate}
\end{definition}

\begin{theorem}[Chaotic Mix Unitarity]
The chaotic mix variant is exactly unitary (by construction).
\end{theorem}

\begin{proof}
The QR decomposition of a random matrix yields a unitary $Q$ factor with probability~1. The diagonal phase adjustment preserves unitarity.
\end{proof}

%============================================================================
\section{Hybrid Basis Decomposition (Theorem 10)}
%============================================================================

\begin{definition}[DCT-II Basis]
The Type-II Discrete Cosine Transform matrix $C \in \R^{n \times n}$ is:
\begin{equation}
C_{km} = \sqrt{\frac{2}{n}} \cos\left(\frac{\pi k (2m+1)}{2n}\right) \cdot \begin{cases} 1/\sqrt{2} & k=0 \\ 1 & k>0 \end{cases}
\end{equation}
\end{definition}

\begin{lemma}[DCT Orthonormality]
$C^\top C = I_n$.
\end{lemma}

\begin{proof}
Standard result from DCT theory (see Oppenheim \& Schafer, 2010). The cosine functions form an orthogonal basis on the discrete grid with appropriate normalization.
\end{proof}

\begin{theorem}[Hybrid Basis Decomposition]\label{thm:hybrid}
For any signal $x \in \C^n$ and any $K_1, K_2 \in \{1,\ldots,n\}$, there exists a decomposition:
\begin{equation}
x = x_{\mathrm{struct}} + x_{\mathrm{texture}} + x_{\mathrm{residual}}
\end{equation}
satisfying:
\begin{enumerate}
    \item $x_{\mathrm{struct}}$ has at most $K_1$ non-zero DCT coefficients
    \item $x_{\mathrm{texture}}$ has at most $K_2$ non-zero RFT coefficients (of the residual)
    \item $\norm{x}^2 = \norm{x_{\mathrm{struct}}}^2 + \norm{x_{\mathrm{texture}}}^2 + \norm{x_{\mathrm{residual}}}^2$ (exact energy split)
\end{enumerate}
\end{theorem}

\begin{proof}
We construct the decomposition and prove the energy identity.

\textbf{Step 1: Define best $K$-term approximation.}
For orthonormal basis $U$ and signal $y$, let $c = Uy$ be the coefficient vector. Define:
\begin{equation}
P_U^{(K)} y = U\dg T_K(Uy)
\end{equation}
where $T_K$ keeps the $K$ largest-magnitude entries and zeros the rest.

\textbf{Step 2: Construction.}
\begin{align}
x_{\mathrm{struct}} &= P_C^{(K_1)} x \quad \text{(best $K_1$-term DCT approx)} \\
r &= x - x_{\mathrm{struct}} \\
x_{\mathrm{texture}} &= P_\Psi^{(K_2)} r \quad \text{(best $K_2$-term RFT approx of residual)} \\
x_{\mathrm{residual}} &= r - x_{\mathrm{texture}}
\end{align}

\textbf{Step 3: Energy identity by Parseval.}
Since $C$ (DCT matrix) is orthonormal, for any vector $y$: $\norm{y}^2 = \norm{Cy}^2$.

Let $c = Cx$. Then $c_S = T_{K_1}(c)$ (the $K_1$ kept coefficients) and $c_{S^c}$ (the zeroed coefficients) are orthogonal:
\begin{equation}
\norm{c}^2 = \norm{c_S}^2 + \norm{c_{S^c}}^2
\end{equation}

Now $x_{\mathrm{struct}} = C\dg c_S$ and $r = C\dg c_{S^c}$, so by Parseval:
\begin{equation}
\norm{x}^2 = \norm{c}^2 = \norm{c_S}^2 + \norm{c_{S^c}}^2 = \norm{x_{\mathrm{struct}}}^2 + \norm{r}^2
\end{equation}

Applying the same argument to $r$ with basis $\Psi$ (which is unitary by Theorem~\ref{thm:unitarity}):
\begin{equation}
\norm{r}^2 = \norm{x_{\mathrm{texture}}}^2 + \norm{x_{\mathrm{residual}}}^2
\end{equation}

Combining:
\begin{equation}
\norm{x}^2 = \norm{x_{\mathrm{struct}}}^2 + \norm{x_{\mathrm{texture}}}^2 + \norm{x_{\mathrm{residual}}}^2 \qquad \qed
\end{equation}
\end{proof}

\begin{remark}
This theorem establishes the \emph{existence} of a sequential decomposition, not optimality over all possible joint decompositions. Global optimality would require solving:
\begin{equation}
\min_{x_s, x_t} \norm{Cx_s}_0 + \norm{\Psi x_t}_0 \quad \text{s.t.} \quad x = x_s + x_t
\end{equation}
which is NP-hard in general (Basis Pursuit Denoising).

\textbf{Implementation Note:} Practical implementations (e.g., H3HierarchicalCascade in algorithms/rft/hybrids/) may use approximate decompositions such as moving average filters for computational efficiency. These approximations do not guarantee exact energy preservation but provide near-orthogonal decompositions with $\norm{x}^2 \approx \norm{x_{\mathrm{struct}}}^2 + \norm{x_{\mathrm{texture}}}^2 + 2\langle x_{\mathrm{struct}}, x_{\mathrm{texture}}\rangle$ where the cross-term is small but non-zero. For applications requiring exact energy accounting, use the orthonormal basis projection method defined in this theorem.
\end{remark}

%============================================================================
\section{Twisted Convolution Algebra}
%============================================================================

\begin{definition}[Golden-Twisted Convolution]
For $x, h \in \C^n$, define:
\begin{equation}
(x \star_{\phi,\sigma} h) = \Psi\dg \left(\mathrm{diag}(\Psi h) \cdot \Psi x\right)
\end{equation}
\end{definition}

\begin{theorem}[Diagonalization of Twisted Convolution]\label{thm:twisted}
\begin{equation}
\Psi(x \star_{\phi,\sigma} h) = (\Psi x) \odot (\Psi h)
\end{equation}
\end{theorem}

\begin{proof}
\begin{align}
\Psi(x \star_{\phi,\sigma} h) &= \Psi \Psi\dg \mathrm{diag}(\Psi h) \Psi x \\
&= \mathrm{diag}(\Psi h) \Psi x \quad \text{(since } \Psi\Psi\dg = I) \\
&= (\Psi x) \odot (\Psi h)
\end{align}
\end{proof}

\begin{theorem}[Commutativity and Associativity]
The twisted convolution $\star_{\phi,\sigma}$ is commutative and associative.
\end{theorem}

\begin{proof}
\textbf{Commutativity:} Since element-wise product is commutative:
\begin{equation}
\Psi(x \star h) = (\Psi x) \odot (\Psi h) = (\Psi h) \odot (\Psi x) = \Psi(h \star x)
\end{equation}
Applying $\Psi\dg$: $x \star h = h \star x$.

\textbf{Associativity:} 
\begin{equation}
\Psi((x \star h) \star g) = ((\Psi x) \odot (\Psi h)) \odot (\Psi g) = (\Psi x) \odot ((\Psi h) \odot (\Psi g)) = \Psi(x \star (h \star g))
\end{equation}
\end{proof}

%============================================================================
\section{Sparsity Properties}
%============================================================================

\begin{definition}[Golden Quasi-Periodic Signal]
A signal $x \in \C^n$ is $K$-golden-quasi-periodic if:
\begin{equation}
x_m = \sum_{j=1}^K a_j \exp\left(2\pi i \cdot \fracpart{j\phi} \cdot m / n\right)
\end{equation}
for some amplitudes $a_j \in \C$.
\end{definition}

\begin{definition}[Sparsity]
The sparsity $S$ of a coefficient vector $y \in \C^n$ with respect to threshold $\tau$ is:
\begin{equation}
S(y, \tau) = \frac{\#\{k : |y_k| < \tau\}}{n}
\end{equation}
i.e., the fraction of coefficients below threshold.
\end{definition}

\begin{conjecture}[Sparsity for Golden Signals]\label{conj:sparsity}
For $K$-golden-quasi-periodic signals with $K \ll n$, the RFT concentrates energy into approximately $K$ coefficients.
\end{conjecture}

\textbf{Intuition:}
The golden phase matrix $D_\phi$ has diagonal entries $\exp(2\pi i \beta \fracpart{k/\phi})$. For a signal whose frequencies are also at golden-ratio positions $\fracpart{j\phi}$, one might expect constructive interference at matching indices.

\textbf{Missing steps for a proof:}
\begin{enumerate}
    \item Precisely characterize which RFT indices receive significant energy (golden resonance analysis)
    \item Show that the golden phase alignment concentrates energy, not just redistributes it
    \item Derive explicit bounds on the energy in non-dominant bins (e.g., exponential decay)
    \item Prove concentration is better than DFT for golden quasi-periodic signals
\end{enumerate}

\textbf{Numerical evidence:}
Empirical tests show sparsity up to 98\% for $N=512$ on synthetic golden-ratio signals, but this may be construction-dependent. For general signals, sparsity improvement over DFT is typically modest (5-15\%).

\textbf{Status:} Plausible conjecture requiring rigorous harmonic analysis; no proof currently exists.

%============================================================================
\section{Summary of Results}
%============================================================================

\subsection{Classification of Results}

\begin{table}[ht]
\centering
\begin{tabular}{llp{4cm}}
\toprule
\textbf{Result} & \textbf{Statement} & \textbf{Status} \\
\midrule
Theorem~\ref{thm:unitarity} & Closed-form RFT unitarity & \textbf{PROVEN} \\
Theorem~\ref{thm:canonical_unitary} & Canonical RFT unitarity & \textbf{PROVEN} \\
Theorem~\ref{thm:complexity} & $O(n \log n)$ complexity & \textbf{PROVEN} \\
Theorem~\ref{thm:hybrid} & Hybrid decomposition with energy identity & \textbf{PROVEN} \\
Theorem~\ref{thm:twisted} & Twisted convolution diagonalization & \textbf{PROVEN} \\
\midrule
Observation~\ref{obs:noneq} & No equivalence found for $n \leq 32$ & Numerical \\
Observation~\ref{obs:alignment} & Canonical $\approx$ closed-form alignment & Numerical \\
\midrule
Conjecture~\ref{conj:nonlct} & Canonical RFT is not an LCT & Open \\
Conjecture~\ref{conj:sparsity} & Sparsity for golden signals & Open \\
\bottomrule
\end{tabular}
\caption{Classification of results: proven theorems, numerical observations, and open conjectures.}
\end{table}

\subsection{Honest Assessment}

\textbf{What is proven:}
\begin{itemize}
    \item The closed-form RFT is unitary and computable in $O(n \log n)$ via FFT
    \item The closed-form RFT is \emph{trivially equivalent} to a phased DFT (Remark~\ref{rem:trivial})
    \item The canonical QR-based RFT is unitary by construction
    \item The hybrid decomposition gives exact energy accounting via Parseval
\end{itemize}

\textbf{What is not proven:}
\begin{itemize}
    \item That the canonical RFT is structurally distinct from the DFT orbit for all $n$
    \item The precise relationship between canonical and closed-form constructions
    \item Any sparsity advantages for specific signal classes
\end{itemize}

\textbf{Note on closed-form RFT:} The closed-form $\Psi = D_\phi C_\sigma F$ is trivially equivalent to a phased DFT ($\Psi = \Lambda_1 F$ with $\Lambda_1 = D_\phi C_\sigma$). Its value lies in computational efficiency, not structural novelty.

%============================================================================
\section{Numerical Observations}
%============================================================================

The following results are purely numerical observations, not theorems. They provide empirical motivation but do not constitute proofs.

\subsection{Quantum Chaos Signature}

\begin{observation}[Level Spacing Statistics]
The eigenvalue level spacing distribution of $\Psi$ exhibits level repulsion consistent with Wigner-Dyson statistics (Gaussian Orthogonal Ensemble), rather than Poisson statistics.
\end{observation}

\textbf{Method:}
\begin{enumerate}
    \item Compute eigenvalues $e^{i\theta_k}$ of $\Psi$
    \item Extract eigenphases $\theta_k \in [0, 2\pi)$
    \item Sort and compute unfolded spacings: $s_k = (\theta_{k+1} - \theta_k)/\langle s \rangle$
    \item Compute variance of $\{s_k\}$
\end{enumerate}

\textbf{Results:}
\begin{itemize}
    \item Variance ratio $\approx 0.26$ (GOE prediction: $\approx 0.273$)
    \item Poisson (uncorrelated levels) would give variance $\approx 1.0$
\end{itemize}

\textbf{Interpretation:}
This suggests RFT has ``mixing'' behavior characteristic of quantum chaotic systems, distinct from the ordered spectrum of the DFT. This is relevant for cryptographic applications but is \emph{not} a proof of any security property.

\subsection{Approximate Equivalence of QR and Closed-Form}

\begin{observation}[QR vs Closed-Form Alignment]
For the canonical QR-derived RFT $U_\phi$ and the closed-form $\Psi = D_\phi C_\sigma F$:
\begin{equation}
\norm{U_\phi\dg \Psi - \Lambda}_F < 10^{-10}
\end{equation}
for some diagonal unitary $\Lambda$.
\end{observation}

\textbf{Interpretation:}
Both constructions appear to produce the same unitary up to column phases. This is numerically observed but not proven analytically.

%============================================================================
\section{Numerical Validation of Proven Results}
%============================================================================

The proven theorems have been validated numerically with the following precision:

\begin{table}[h]
\centering
\begin{tabular}{lll}
\toprule
\textbf{Property} & \textbf{Error Bound} & \textbf{Test Sizes} \\
\midrule
Unitarity $\norm{\Psi\dg\Psi - I}_F$ & $< 10^{-14}$ & $n \in \{32, 64, 128, 256, 512\}$ \\
Round-trip $\norm{\Psi\dg\Psi x - x}/\norm{x}$ & $< 10^{-14}$ & 1000 random vectors \\
Energy preservation & $< 10^{-14}$ & 1000 random vectors \\
Twisted convolution identity & $< 10^{-15}$ & 100 random pairs \\
Non-quadratic $\Delta^2 f$ & Exact: $\{-1, 0, 1\}$ & Direct computation \\
\bottomrule
\end{tabular}
\caption{Numerical validation of proven theorems}
\end{table}

%============================================================================
\section*{References}
%============================================================================

\begin{enumerate}
    \item Oppenheim, A.V., Schafer, R.W. (2010). \textit{Discrete-Time Signal Processing}, 3rd ed. Pearson.
    \item Cooley, J.W., Tukey, J.W. (1965). ``An Algorithm for the Machine Calculation of Complex Fourier Series.'' \textit{Math. Comp.} 19:297--301.
    \item Wolf, K.B. (1979). \textit{Integral Transforms in Science and Engineering}. Plenum Press.
    \item Haake, F. (2010). \textit{Quantum Signatures of Chaos}, 3rd ed. Springer.
\end{enumerate}

\end{document}
